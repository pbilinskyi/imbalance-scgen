\setcounter{secnumdepth}{2}
\section{Вступ}
\subsection{Актуальність}
Персоналізоване лікування - нова парадигма лікування хвороб, котра передбачає складання індивідуального плану лікування для кожного пацієнта на основі унікальних біологічних характеристик його організму. Мотивацією для розвитку персоналізованого лікування є суттєві недоліки традиційного підходу, котрий практикується у більшості лікувальних установ світу. Зазвичай пацієнти зі схожими випадками отримують схожі рекомендації лікаря, при цьому індивідуальні особливості організму пацієнта не враховуються. Таке лікування може бути неефективним для конкретного пацієнта, або ж навіть нашкодити йому. За даними досліджень, найперше призначене лікування є неефективим у більш ніж $50\%$ випадків захворювання артритом, хворобою Альцгеймера або раком [Spear]. Деякі клініки вже пропонують послуги “молекулярної діагностики” для пацієнтів із лейкемією, раком легень, грудей чи головного мозку, і це дає можливість підібрати інтивідуальний план лікування, що значно підвищує шанси на виживання. Персоналізоване лікування - це майбутнє медицини, і зараз в цій сфері ведуться інтенсивні дослідження.

Для втілення ідеї персоналізованого підходу в життя потрібно могти передбачати реакцію клітин організму на ті чи інші типи медикаментів. З появою великої бази генетичних даних та розвитком методів машинного навчання відкрилися нові можливості для побудови точних предиктивних моделей. Найперші моделі були механістичнимим, тобто, грунтувались на певній математичній моделі, що описує молекулярні процеси [scGen]. Значно ефективнішими виявилися моделі, навчались з даних, використовуючи методи машинного навчання [scGen]. Важливо було досягти хорошої точності передбачення не лише на тренувальній вибірці, але й мати здатність до узагальнення. Тут найбільшого успіху досягла модель scGen, побудована на основі варіаційного автоенкодера. Ця модель будує якісні представлення даних в просторі низької розмірності, що забезпечує хорошу здатність до узагальнення, а також можливість генерувати нові дані для чисельних експериментів.

Проблема дисбалансу класів - важлива проблема для багатьох моделей машинного навчання, котра також актуальна і для задачі передбачення збурення клітини. Недавнє дослідження показало важливість проблеми дисбалансу даних в області аналізу одноклітинних даних, зокрема для задачі одноклітинної інтеграції [hassIntegr]. Однак його ще належить дослідити для задачі прогнозування клітинних збурень. Оскільки процес отримання нових даних - дороговартісний, важливо максимально використати потенціал існуючої вибірки. В контексті задачі передбачення пертурбацій клітини проблема все ще не була досліджена, і в цьому полягає новизна роботи.

\subsection{Мета, завдання дослідження}
В цій роботі досліджується вплив дисбалансу класів в даних на точність перебачень моделі scGen. Порівнюються методи вирішення проблеми дисбалансу.


Це передбачає розгляд таких задач:


\begin{itemize}

\item Доповнити модель scGen механізмом Attention для побудови механізму пояснення результатів передбачення.

\item Застосувати методи боротьби з проблемою дисбалансу даних до задачі передбачення збурень клітин та оцінити вплив на якісь передбачень.

\item Дослідити вплив фактору однорідності на якість передбачень.

\item Знаходження оптимального за якістю передбачення методу боротьби з дисбалансом у даних.

\end{itemize}