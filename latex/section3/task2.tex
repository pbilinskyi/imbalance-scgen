\subsection{Загальне відновлююче спектральне число $srn(G)$}
Введемо деякі означення.

 \textbf{Означення 14.}\\
 \textit{Загальне відновлююче спектральне число $srn(G)$} --- мінімальна кількість спектрів підграфів, що були утворені видаленням ребер, за якими однозначно відновлюються ваги ребер вихідного графа. Будемо позначати з малої літери.\\
 
 Отже, для наступних графів розглянемо задачу, що буде мати такі дві підзадачі: навести приклад такого набору підспектрів, за яким можна відновити усі ваги, та знайти $srn({G})$. Таку задачу будемо називати загальною.
