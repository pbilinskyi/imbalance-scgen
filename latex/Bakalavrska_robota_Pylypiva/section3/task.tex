Сформулюємо постановку задачі, яку надалі будемо розглядати у цьому розділі та наступному розділі.

\textbf{Постановка задачі.}\\
Нехай нам відомий довільний граф $G$. Ми хочемо однозначно відновити ваги кожного ребра зваженого графа ${\bf G} = (G,w)$, тобто його вагову функцію $w$, за спектрами його підграфів.

\textbf{Лема 1.}\\
Відновлення ваги для кожного ребра зваженого графа \textbf{G} за спектрами його підграфів і відновлення за характеристичними многочленами цих підграфів є еквівалентними задачами.

\textit{Доведення.}\\
Характеристичний многочлен $P_{\bf G}$ довільного зваженого графа \textbf{G} можна однозначно відновити за спектром і навпаки, бо корені многочлена, у якого старший коефіцієнт дорівнює одиниці, однозначно його відновлюють.

Нехай $\sigma({\bf G}) = \{\lambda_1,\lambda_2, ...,\lambda_n \}$ --- спектр зваженого графа \textbf{G}, то
\begin{equation}
    P_{\bf G}(\lambda) = (\lambda - \lambda_1)(\lambda - \lambda_2)\cdot...\cdot(\lambda - \lambda_n)= \prod_{i=1}^n(\lambda - \lambda_i)
\end{equation}

\subsection{Відновлююче спектральне число $Srn(G)$}
Розглянемо першу задачу і введемо деякі означення.

\textbf{Означення 11.}\\
\textit{Породжений або індукований підграф графа} {\it G} --- підграф, утворений підмножиною вершин графа {\it G} і усіма ребрами, що з'єднують ці вершини.

\textbf{Означення 12.}\\
{Підспектр графа \it G} --- спектр  підграфа. 

\textbf{Означення 13.}\\
\textit{Відновлююче спектральне число} $Srn({G})$ --- мінімальна кількість спектрів породжених підграфів, за якими однозначно відновлюються ваги ребер вихідного графа.

Нижня границя $Srn({G})$ дорівнює 1, тобто спектр вихідного графа відновлює свої ж ваги. Єдиним прикладом такого графа є ${\bf A_2}$, тобто ребро, спектр якого дорівнює $\sigma({\bf A_2})=\{-w,w\}$ і однозначно відновлює вагу на ребрі. Отже, для довільного графа відмінного від ${\bf A_2}:$ $Srn({G})\geq2$.

Верхня границя для довільного графа \textbf{G}: $Srn({G})$ дорівнює $n$, бо будь-який зважений граф можна відновити знаючи спектри підграфів на двох вершинах, тобто ребер.\\
Тобто $ 1 \leq Srn({G}) \leq n$.\\

Отже, спершу для наступних графів розглянемо задачу, що буде мати такі дві підзадачі: навести приклад такого набору підспектрів, за яким можна відновити усі ваги, та знайти $Srn({G})$.
