\subsection{Основні означення з спектральної теорії зважених графів}

 \textbf{Означення 6.}\\
\emph{Матриця суміжності} --- це квадратна матриця:
\begin{equation}
    {A({\bf G})}=\|a_{ij}\|_{i,j=1}^{n},
\end{equation}

де $n=|{ G}|$ --- кількість вершин у графі,

 $a_{ij}=a_{ji}$ --- елемент матриці, що дорівнює 0, якщо вершина $i$ та $j$ несуміжні, або $w_{ij}$, якщо вершини з'єднані ребром. 

Тобто матриця суміжності є симетричною, оскільки $a_{ij}=a_{ji}$, і вона також має нулі по всій діагоналі, бо немає петель.

 \textbf{Означення 7.}\\
 \textit{Точки спектра} --- це власні значення матриці суміжності.
 
 Оскільки матриця суміжності симетрична, то її спектр буде дійсним.
 
 \textbf{Означення 8.}\\
{\itСпектр} графа $\sigma({\bf G})$ --- спектр матриці суміжності. Від зміни нумерації вершин спектр не змінюється.

Для характеристичного многочлена матриці суміжності зваженого графа будемо використовувати таке позначення : 
\begin{equation}\label{eq1}
    P_{\bf G}(\lambda)=|\lambda\textit{I}-A(\bf G)|.
\end{equation}

