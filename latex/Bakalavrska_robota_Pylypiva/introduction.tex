\setcounter{secnumdepth}{2}
\section{Вступ}
\subsection{Актуальність}
Спектральна теорія графів --- розділ математики, що займається дослідженням властивостей графів за власними значеннями його матриці суміжності, тобто його спектру.

Цей розділ виник у середині 20 століття. У 1980 роках у монографії D. Cvetkovic, M. Doob та H. Sachs[1] узагальнили результати у цій області, що були вже отримані на той час. Трохи пізніше були також видані роботи, що описували підсумки нових досліджень. Детальніше з ними можна ознайомитися у [2] та [3].

Розвиток спектральної теорії графів не стоїть на одному місці, вона доволі активно розвивається, оскільки широко застосовується у багатьох науках, наприклад у хімії, біології, фізиці, математиці та інших. У комп’ютерній науці спектри використовуються при розпізнаванні образів, комп’ютерному зорі та інших інтернет-технологіях. У книзі D. Cvetkovic та I. Gutman[4] та у статтях [5], [6] розглянуто різноманітні застосування у різних галузях.

У своїй роботі я розглядаю саме обернені спектральні задачі, тобто за спектром графа та підграфів відновлюємо ваги на його ребрах. До таких задач також чи мало уваги, наприклад G.M.L. Gladwell у своїй роботі [7] розглядає реконструкцію вібраційної системи, що пов’язано з оберненими спектральними задачами.

Тобто тема кваліфікаційної роботи є досить актуальною. Вона активно розвивається, є багато можливостей для застосувань, а також задачі, які  можна дослідити та отримати нові результати.

\subsection{Мета, завдання дослідження}
Метою кваліфікаційної роботи є дослідження зв’язку спектрів зважених графів і їх підграфів для відновлення вагової функції, тобто ваги кожного ребра вихідного графа.

Це передбачає розгляд таких задач:
\begin{itemize}
\item Знаходження відновлюючого спектрального числа для циклу на трьох вершинах.
\item Знаходження відновлюючого спектрального числа для циклу на чотирьох вершинах.
\item Знаходження підспектрів, за якими  можна відновити граф метелик.
\item Знаходження верхньої оцінки відновлюючого спектрального числа для уніциклічних графів.
\item Знаходження верхньої оцінки відновлюючого спектрального числа для графів кактусів-ланцюгів.
\item Загальна обернена задача для графів, що містять $2k$ непарних вершин.
\item Загальна обернена задача для графів-циклів.
\item Знаходження верхньої оцінки загального відновлюючого спектрального числа для графів кактусів.
\end{itemize}