\subsection{Знаходження характеристичного многочлена та визначника матриці суміжності}
Введемо деякі означення та позначення для теорем.

\textbf{Означення 9.}\\
 {\it Лінійний підграф графа ${G}$} --- підграф, компонентами зв'язності якого є тільки ребра та цикли. Позначимо через $H_k$, де k = $|H_k|$, тобто дорівнює кількості вершин.
 
 \textbf{Означення 10.}\\
 {\it Каркасний підграф графа ${G}$} --- лінійний підграф, що містить усі вершини вихідного графа. 

Введемо деякі позначення, які надалі будемо використовувати у теоремі Харарі та Захса:\\
-- $p(H_{k})$ --- кількість компонент зв'язності лінійного підграфа $H_{k}$, що мають парну кількість вершин.\\
-- $r(H_{k})$ --- кількість компонент зв'язності лінійного підграфа $H_{k}$.\\
-- $c(H_{k})$ --- кількість компонент зв'язності лінійного підграфа $H_{k}$, що є циклами.\\
-- $w(H_{k})$ --- вага $H_{k}$, яка є добутком усіх ваг його
компонент зв'язності. Якщо компонента зв'язності це ребро
$(i,j)$, то його вага буде дорівнювати $w_{i j}^2$, а якщо цикл, то  вага дорівнює добутку значень $w_{i j}$ по всіх ребрах $(i,j)$.\\

\textbf{Теорема 1 (Харарі [11]). }\\
 Визначник матриці суміжності довільного зваженого графа ${\bf G} =(G,w)$ можна порахувати за такою формулою:
 \begin{equation}\label{eq2}
     \det A({\bf G})=\sum_{\{H_{n}\}}(-1)^{p(H_{n})}2^{c(H_{n})}w(H_{n})
 \end{equation}
 
\textbf{Теорема 2 (Захса [12]). }\\ 
Якщо $P_{\bf G}(\lambda)={\sum_{k=0}^{n}}c_{k}\lambda^{n-k}=\lambda^n+c_{1}\lambda^{n-1}+c_{2}\lambda^{n-2}+...+c_{n}$,---
характеристичний многочлен графа ${\bf G} =(G,w)$, то


(1) $c_{1}=0$;

(2) $c_{2}=-\sum_{e\in E({\bf G})} {w(e)}^2$

(3) $c_{k}=\sum_{\{H_{k}\}}(-1)^{r(H_{k})}2^{c(H_{k})}w(H_{k})$ для
$k=1,...,n.$\\

Також запишемо теорему, що узагальнює теорему Швенка[13], показує співвідношення між спектром зваженого графа \textbf{G} і спектром його підграфа, тобто між їх характеристичними многочленами.

Також порожній граф для зручності будемо позначати $K_0$ і його характеристичний многочлен $P_{\bf K_0}(\lambda) = 1$. 

\textbf{Теорема 3.}\\
Нехай $v$ — вершина зваженого графа $G$, через $C(v)$ позначимо множину циклiв, що мiстять $v$. Тодi
\begin{equation}
    P_{\textbf{G}}(\lambda) = \lambda P_{\textbf{G}-v}(\lambda)-\sum_{u~v}w^2_{u v}P_{\textbf{G}-v-u}(\lambda) - 2\sum_{Z\in C(v)}w(Z)P_{\textbf{G}-V(Z)}(\lambda)
\end{equation}

\textbf{Наслідок 1 (Розклад за висячою вершиною).}\\
Якщо вершина $v$ --- висяча вершина графа \textbf{G} і $u$ та $v$ --- суміжні вершини, то
\begin{equation}
    P_{\textbf{G}}(\lambda) = \lambda P_{\textbf{G}-v}(\lambda)-w^2_{u v}P_{\textbf{G}-v-u}(\lambda) 
\end{equation}