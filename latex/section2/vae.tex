\subsection{Варіаційний автоенкодер}

{
\color{red}
Варіаційний автоенкодер (VAE) - це тип генеративної моделі, яка може навчитися створювати нові дані, схожі на навчальні дані. Це тип нейронної мережі, яка складається з мережі енкодера та мережі декодера.

Мережа енкодера отримує на вхід дані та виробляє латентний вектор, який представляє стисле представлення вхідних даних. Мережа декодера отримує латентний вектор та відтворює початкові дані.

Однак, основна різниця між VAE та традиційним автоенкодером полягає в тому, що латентний вектор, який виробляється мережею енкодера, не є фіксованим вектором. Замість цього, це є розподіл ймовірностей над можливими значеннями латентного вектора. Це означає, що кожен вхід відображається на розподіл над можливими латентними векторами, а не на один вектор.

Під час навчання VAE намагається навчитися параметрів розподілу ймовірностей, які найкраще апроксимують справжній розподіл латентних векторів для вхідних даних. Це робиться шляхом мінімізації функції втрат, яка вимірює різницю між відтвореним вхідним значенням та початковим значенням, а також різницю між вивченим розподілом та попередньо визначеним розподілом (зазвичай стандартним нормальним розподілом).
}