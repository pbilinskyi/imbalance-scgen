\thispagestyle{empty}
 \begin{center}
 Міністерство освіти і науки України\\
 Національний університет «Києво-Могилянська 
академія»\\
Факультет інформатики\\
Кафедра математики
\end{center}

\begin{flushright}


ЗАТВЕРДЖУЮ\\
Зав.кафедри математики,\\
проф., доктор фіз.-мат. наук\\ 
$\underset{\text{(підпис)}}{\underline{\hspace{3.6cm}}}$\emph{Олійник Б.В.}\\

\begin{tabular}{l}
 ``\noindent\rule{1cm}{0.4pt}''\noindent\rule{3.9cm}{0.4pt} 2023\\
\end{tabular}

\end{flushright}
\vspace{1mm}


\begin{center}
ІНДИВІДУАЛЬНЕ ЗАВДАННЯ\\
для курсової роботи  \\
студенту 5-го курсу, факультету інформатики\\
Білінському Павлу Олександровичу
\end{center}
\noindent{\bf Тема:} «Редукція розмірності табулярних вхідних даних в контексті обробки одноклітинних даних»\\
\noindent{\bf Зміст курсової роботи:}\\
\hspace*{13mm}
{ \color{red}

\begin{tabular}{l}
 Анотація\\
 1. Вступ\\
 2. Огляд основних означень та тверджень, що пов'язані\\ зі спектральною теорією графів\\
 3. Розв’язання обернених спектральних задач для різних типів\\ зважених графів\\
 4. Розв’язання загальних обернених спектральних задач\\
 Висновки\\
 Список літератури \end{tabular}
}

\vspace{5mm} 
\begin{flushright}
 
 \begin{tabular}{clclc} 
  \noindent Дата видачі ``\noindent\rule{1cm}{0.4pt}''\noindent\rule{2.5cm}{0.4pt} 2021 & Керівник  $\underset{\text{(підпис)}}{\underline{\hspace{3.6cm}}}$\\
  
  & Завдання отримав  $\underset{\text{(підпис)}}{\underline{\hspace{3.6cm}}}$\\
 \end{tabular}
\end{flushright}
